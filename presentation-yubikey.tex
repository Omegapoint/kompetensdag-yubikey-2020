\documentclass{beamer}

\usetheme[
  titleformat=smallcaps,
  numbering=fraction]{metropolis}

\setmonofont{Fira Mono}[BoldFont={Fira Mono Medium},Scale=MatchLowercase]

\usepackage{fontawesome}

\usepackage{minted}
\usemintedstyle{vs}

\usepackage{polyglossia}
\setdefaultlanguage{swedish}

\usepackage{hyperref}


\title{Säkrare SSH med Yubikey}
\date{2020-02-07}
\author{Viktor Ahlqvist}
\institute{Omegapoint kompetensdag}

\begin{document}
\maketitle

\begin{frame}{Målbild}
  \begin{itemize}
    \item Inga nycklar på datorn
    \item Möjlighet att ta med nycklarna
    \item Möjliggör lättare backup, skriptad installation, …
  \end{itemize}
\end{frame}

\begin{frame}{Open PGP nycklar}
  \begin{itemize}
    \item Master key
      \begin{itemize}
        \item Skapa nycklar
        \item Signera/certifiera nycklar
        \item Visa att du äger nycklar
      \end{itemize}
    \item Subkeys
      \begin{itemize}
        \item Används för signering, kryptering och/eller authentisering
        \item Nycklarna som används vardagligen
      \end{itemize}
    \item Nycklar kan genereras på Yubiken eller tas in
  \end{itemize}
\end{frame}

\begin{frame}[fragile]{Konfigurera Yubikeyn}
  \begin{itemize}
    \item Konfigurera med Yubikey manager (ykman)
    \item Stäng av OTP\\
      \mintinline{shell}{$ ykman config usb --disable OTP}
    \item Aktivera touch\\
      \mintinline{shell}{$ ykman openpgp set-touch aut on}
    \item Byt pin\\
      \mintinline{shell}{$ gpg --change-pin}
  \end{itemize}
\end{frame}

\begin{frame}[fragile]{Generera nycklar}
  \begin{columns}
    \begin{column}{0.6\linewidth}
      \begin{itemize}
        \item Generera på Yubikey eller för in nyckel\\
          {\small\emph{Omöjligt (?) att föra ut privat nyckel}}
        \item Generera nycklar\\
          \texttt{generate}
        \item Om intresse: byt nyckeltyp/längd\\
          \texttt{key-attr}
      \end{itemize}
    \end{column}
    \begin{column}{0.4\linewidth}
      \begin{minted}[autogobble, fontsize=\small]{shell}
        $ gpg --card-edit
        Reader ...........: 1050:04
        …
        gpg/card> admin
        …
        gpg/card> key-attr
        …
        gpg/card> generate
      \end{minted}
    \end{column}
  \end{columns}
\end{frame}

\begin{frame}[fragile]{För SSH: GPG agent}
  \begin{itemize}
    \item GPG agenten för SSH agenten
    \item \emph{pinentry} program för att mata in pinkoden
    \item \mintinline{shell}{$ ssh-add -L} bör visa en \texttt{cardno:}-nyckel
    \item Ladda upp på Github (eller motsv)
    \item Testa med \mintinline{shell}{$ ssh -v git@github.com}
  \end{itemize}
\end{frame}

\begin{frame}[fragile]{Signera commits i Git}
  \begin{itemize}
    \item Lista kända nycklar\\
      \mintinline{shell}{$ gpg --list-secret-keys --keyid-format LONG}
    \item Exportera publika nyckeln\\
      \mintinline{shell}{$ gpg --armor --export KEYID}
    \item Ladda upp på Github (eller motsv)
    \item Konfigurera git att använda en specifik nyckel\\
      \mintinline{shell}{$ git config --global user.signingkey KEYID}
    \item Signera commits med \texttt{-S} och taggar med \texttt{-s}\\
      \mintinline{shell}{$ $ git commit -S -a -v}\\
      \mintinline{shell}{$ git tag -a '1.0.0' -m 'First release' -s}
  \end{itemize}
\end{frame}

\begin{frame}{Källor}
  \begin{itemize}
    \item drduhs Yubikey-Guide https://github.com/drduh/YubiKey-Guide/
    \item Debians information om subkeys: https://wiki.debian.org/Subkeys
    \item esev https://www.esev.com/blog/post/2015-01-pgp-ssh-key-on-yubikey-neo/
    \item florin (Mac OS) https://florin.myip.org/blog/easy-multifactor-authentication-ssh-using-yubikey-neo-tokens
    \item ixdy (tillägg till florin) https://gist.github.com/ixdy/6fdd1ecea5d17479a6b4dab4fe1c17eb
  \end{itemize}
\end{frame}

% \begin{frame}{}
%   \begin{itemize}
%     \item
%   \end{itemize}
% \end{frame}

\end{document}
